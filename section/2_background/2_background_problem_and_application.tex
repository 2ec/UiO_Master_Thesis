\label{sec:2_problem_and_application}
\section{Problem and application}
% Intro to this chapter

The increasing use of \gls{ai} has led to significant advances in various areas. However, understanding their decision-making processes becomes increasingly difficult as AI systems become more complex and opaque. 
The need for \gls{xai} and more transparent machine learning models has become imperative to address this problem, especially in applications where the consequences of wrong decisions can be severe, such as healthcare, finance, and autonomous vehicles. This chapter aims to provide an overview of the problem and application of \gls{xai}, which is crucial for building transparent and trustworthy \gls{ai} systems.


% Problem:
\subsection{Problem}
% ------------------------------------------

% Black-box
Although deep networks have made significant positive achievements in areas such as object detection \cite{girshickRichFeatureHierarchies2014, renFasterRCNNRealTime2015, redmonYouOnlyLook2016, linFocalLossDense2017}, image annotation and captioning \cite{vinyalsShowTellNeural2015, karpathyDeepVisualSemanticAlignments2015, johnsonDenseCapFullyConvolutional2016, tranRichImageCaptioning2016}, their complexity makes it more difficult to understand why these networks predict what they do. People, both researchers and users, of these systems need to be able to know how these black-box algorithms work to gain confidence \cite{koehlerExplanationImaginationConfidence1991, herlockerExplainingCollaborativeFiltering2000, dzindoletRoleTrustAutomation2003}, improve the models and apply these networks in new ways and domains \cite{jiangArtificialIntelligenceHealthcare2017, tonekaboniWhatCliniciansWant2019, holzingerCausabilityExplainabilityArtificial2019, guptaDeepLearningObject2021, tjoaSurveyExplainableArtificial2021}. Researchers and model architects can understand at a higher level how information flows in the network compared to everyday users by using their technical insight. However, as the architecture grows more profound and complex, and the training datasets are getting larger, it can be challenging to understand which parts of the input data contributed to making the decision \cite{sagirogluBigDataReview2013}.

% Interpretablity vs. accuracy 
\paragraph{Interpretablity vs. Accuracy\\}
The models with the highest accuracy for large data sets often have such complex architectures that even domain experts have difficulty interpreting their decisions \cite{caruanaIntelligibleModelsHealthCare2015}. Meanwhile, smaller, less complex architectures often have lower accuracy and generalize worse than their more complex counterparts. An example of this was shown in a case study to predict the risk of death for patients with pneumonia to help medical staff prioritize \cite{cooperPredictingDireOutcomes2005}. The researchers found the most accurate model to be a neural network, outperforming less complex models such as logistic regression. A rule-based system was also evaluated, and while this is a simpler model than neural networks, it is interpretable by design. This rule-based model to investigate the underlying dataset showed that a patient suffering from pneumonia and asthma had a lower probability of dying than only having pneumonia and was, therefore, less important to treat. The model drew this conclusion because patients in the training set with both pneumonia and asthma were usually prioritized first, was given medical treatment, and therefore had a higher survival rate \cite{cooperEvaluationMachinelearningMethods1997}. Because of this insight from the rule-based method, more complex models, such as neural networks, were concluded to be too risky in real-world decisions.

In pursuing models with higher accuracies, the primary way to achieve this is with even more complex models and larger training datasets \cite{bianchiniComplexityNeuralNetwork2014}. This brings the trade-off of an interpretable and explainable model vs. a more accurate model \cite{barredoarrietaExplainableArtificialIntelligence2020} to the forefront of discussion. 


% XAI
\paragraph{Explainable AI\\}
The field of \gls{xai} is working on solving the trade-off between performance and explainability. Some approaches specialize in explaining specific architectures, called model-specific. Meanwhile, others, called model-agnostic, try to explain models of different architectures, exploiting inherent properties in neural networks and statistics. Examples of inherently explainable models include decision trees, Bayesian classifiers, logistic regression, linear models, and \gls{knn} \cite{fixDiscriminatoryAnalysisNonparametric1989, coverNearestNeighborPattern1967, molnarInterpretableMachineLearning}. These algorithms are interpretable and, as a result of this, more explainable by design. This interpretability results from their internal structure, and computations follow clear rules or formulas that are manageable to comprehend.
However, models such as deep neural networks can work better than less complex methods on larger datasets. Because of their complex structure, with many hidden layers and weights trained on large datasets, it is difficult, if not impossible, for humans to understand what the model evaluated when choosing a prediction. Researchers in \gls{xai} have developed model-specific and \textit{post-hoc} model-agnostic techniques to understand better these complex methods to explain the underlying model prediction. The term post-hoc comes from Latin and means \textit{"after this"}. In this thesis, the term is used to describe a method applied after a machine learning model is developed. These methods try to bridge the gap between high accuracies and explainability.
In theory, this allows us to use the model best fit for the task, regardless of complexity, without the expense of not understanding the model's predictions. 

% Many models 
Different explanations try to give additional insights into the predictions. A local explanation looks at one specific model decision and tries to explain what was important in the input data to provide this prediction. A global explanation, on the other hand, looks at the whole model's process of making decisions and sees how the different attributes contribute to making a decision. These global explanations can also be built from multiple local faithful explanations. A third type is contrastive explanations that utilize local features to explain the difference between instances. This allows insight into the model's inner workings on a more global scale based on local predictions.

% Cons
One typical disadvantage of model-agnostic explanations is that their insights and descriptions are not always faithful to the underlying model they try to explain. This can happen if an explaining model is trained to look at the underlying model's input, inner workings, and output and learns the correlation between them. The problem is that correlation does not imply causation, and the explaining model can give a deceiving explanation that can look correct at first. Developing explanatory methods that find causations rather than correlations is an ongoing research topic.




% Application (where my system will go):
\subsection{Application}
% ------------------------------------------

% Intro
With better explanations that are intuitive and faithful to the underlying prediction model, humans can be more precise when improving the model. This also gives the ability to use the model with confidence that the prediction is based on correct decisions.


% Medical / bank
When machine learning methods are utilized as a tool in the real world, it is essential that everyone in the process of using the tool can trust the model. 
In a medical setting, both the doctor and the patient must have confidence that the conclusions drawn are based on a reasonable and trustworthy basis. If the model is correct, but the clinician does not trust the underlying model or the patient has no explanation for the model's conclusion, the model is not being used as intended and is, therefore, a useless tool.

This is also the case in other domains, such as finance. Here the bank can utilize large models that look at the loan applicant using big data. The bank can profile the customer alongside fellow citizens in the same demographic and use a model to predict whether or not that customer should receive a loan. In this case, finding all biases in the data set can be difficult, as it can be challenging to distinguish correlation from causation in demographic analysis \cite{garciaHarmsDemographicBias2019}. Here the bank must have an explanation alongside the predicted output of the loan to see better if the model made a trustworthy decision. 

% Non-experts improve models



% Finding bias in datasets
One advantage of better understanding the model's evaluations in a prediction is that it can highlight biases in the dataset. If these biases are known and understood, they can be combated. Ribeiro et al. \cite{ribeiroWhyShouldTrust2016} showed that it could be hard to discover biases when the predictions are correct without knowing the reasoning. They experimented with a model trained on images of huskies and wolfs and first presented the model's prediction without explanations to participants. The participants were then asked to determine if they trusted the model. Thereafter the model's explanation for the predictions was presented, and participants again had to tell if they trusted the model now. In both instances, the participants were also asked if they thought snow was a potential feature of importance. An image of a husky misclassified as a wolf can be seen in Figure \ref{fig:wolf_husky}, alongside the regions important in making the decision. The results can be seen in Table \ref{table:husky_vs_wolf}, and it can be seen that a considerable amount of the participants lost trust in the biased model when they were presented with the explanation. They then noticed that the decision was based only on whether the snow was visible in the image. This shows that models with an intuitive explanation are more likely to gain the trust of their users. Trustworthy explanations open the ability for users who do not have insight into the making of the model to still able to detect biases in the dataset and improve the model. 

\begin{table}[htb]
    \centering
    \begin{tabular}{ c c c} 
     
               & Before explanation & After explanation\\ [0.5ex] 
        \Xhline{1.5pt} \\ 
            Trusted the biased model & 10 out of 27 & 3 out of 27 \\  [0.5ex]
        \hline  \\ 
            Thought snow was an \\ important feature & 12 out of 27 & 25 out of 27 \\ [1ex] 
        \hline
    \end{tabular}
    \caption[Overview over the trust by participants in the "Husky vs. Wolf" experiment by Ribeiro et al. \cite{ribeiroWhyShouldTrust2016}.]{Overview over the trust by participants in the "Husky vs. Wolf" experiment by Ribeiro et al. \cite{ribeiroWhyShouldTrust2016}. The majority lost trust when explained that a classifier considered snow in the background the most important feature when classifying images of huskies and wolves.}
    \label{table:husky_vs_wolf}
\end{table}

\begin{figure}[htb]
    \centering
    \includegraphics[width=\linewidth]{images/husky_vs_wolf.png}
    \caption[Husky classified as a wolf, alongside an explanation of what the model considered important.]{Husky classified as a wolf, alongside an explanation of what the model considered important.\\ Figure by Ribeiro et al. \cite{ribeiroWhyShouldTrust2016}.}
    \label{fig:wolf_husky}
\end{figure} 


% Make humans able to learn from the models
Today's machine learning models are already better than humans in domain-specific tasks, like chess \cite{campbellDeepBlue2002}. Silver et al. \cite{silverGeneralReinforcementLearning2018} proposed the reinforcement learning algorithm AlphaZero, which learns to play chess, shogi (Japanese chess), and Go, only by playing with itself. This method does not get any domain knowledge except the game rules and achieves performance better than humans. Because this algorithm has not seen humans play these games, it does not have human biases and playing flaws in it, and professional game players can learn new moves and techniques by playing with it. 
As performance better than humans is a goal many researchers strive for when it comes to machine learning, it is more important than ever that these algorithms give explanations on what they do and why so that humans can learn new aspects never before thought of, as well as detect flaws that restrict performance, both in humans and machines.
