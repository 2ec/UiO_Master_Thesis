\label{sec:abstract}

\begin{comment}
The abstract is often the first text the reader looks at. Thus, it should be very well written, concise and to-the-point - with the focus of SELLING your work. It is therefore often written at the very end, when you have all details of your work. Usually, we recommend something like this: i) A sentence or three about the background of the challenge you are addressing, which then leads to your “problem statement” (written as just a sentence); ii) Some text describing what you have done in your research, what have you developed, etc.; iii) A small overview of your main results and conclusions - what are the main take-aways from your thesis.

Note also that the abstract is a teaser, and it should therefore not be too long: Fast to read, fast to get to the point. We often recommend keeping it on one page.
\end{comment}


% More complex models
With the increase in accuracy and usability of deep neural networks, there has been a big demand for using these neural networks in different domains to increase productivity, create new industries, and improve people's lives. 
However, these networks are often large and complex, which does not give insight into the prediction process done by the model. 
In order to make the models more functional and be able to improve them, humans need to understand how they reason. 

This thesis will explore image captioning systems that are locally faithful to the underlying method using gradient activations in the network combined with a language model trained on open-ended questions and answers.