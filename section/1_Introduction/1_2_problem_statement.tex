\label{sec:1_2_problem_statement}

\begin{comment}
In a short and precise way, state what your research is about in this thesis. It can be in the form of a (set of) research questions, goals/aims, or objectives (or a mix) - but it should clearly state what the problems or challenges you are addressing.

Alternatively, one can state a research hypothesis, but if so, it should follow the rules of what a hypothesis is. A hypothesis is a statement that introduces a research question and proposes an expected result. It is an integral part of the scientific method that forms the basis of scientific experiments. Therefore, you need to be careful and thorough when building your hypothesis, following the “rules”.
\end{comment}

\section{Problem statement}

% Intro
For machine learning and computer vision to be truly trustworthy, they will need to be understood. Researchers need to understand the inner workings of the model to improve it and discover biases in the dataset. Users and domain experts utilizing the models need to have trust in the model predicting accurately while using useful features when evaluating. The models should be able to achieve \gls{sota} performance, without sacrificing interpretability and explainability in what they evaluate when predicting. 

% What this is and How I want to do it
In this thesis, the goal is to explore the roam of visual and linguistic understanding.
The visual knowledge will come from a \gls{cnn} that learn features in images. An \gls{rnn} \cite{rumelhartLearningRepresentationsBackpropagating1986, choLearningPhraseRepresentations2014, sutskeverSequenceSequenceLearning2014, bahdanauNeuralMachineTranslation2016} or transformer is used to extract linguistic information from open-ended questions and answers in natural language, taken from a \gls{vqa} dataset. A backward pass that looks for large gradients through layers of the \gls{cnn} will be used to choose locally faithful words to caption the image in an explainable way. 



% Questions to be answered
The main topics that will be explored to reach the research goal are: 
\begin{itemize}
    \item Does an explainable image caption model improve its answers when trained on a \gls{vqa} dataset?
    \item With only the explaining model choosing the captions, will the output be more intuitive for humans?
    \item Are the explanations, both captions and answers from open-ended questions locally faithful to the underlying \gls{cnn} model?
    \item If there is enough time during this thesis, it would be an interesting task to see if the image caption and \gls{vqa} will perform better when using transfer learning  on a language model pretrained on a large language dataset. 
\end{itemize}

% What is the result
With these research questions answered this thesis will be one step closer to an explanatory model that can give insight into how models understand broad concepts in vision and language. This knowledge can be used to make computers understand a more complex worldview and help humans understand what it sees and value.

