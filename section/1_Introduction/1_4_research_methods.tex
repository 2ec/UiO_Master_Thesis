\label{sec:1_4_research_methods}

\begin{comment}
You are doing a research education, so it might be nice to show that you are aware of different ways of doing research. Here, you should describe your research methods showing the reader that you are conscious of your method. There exist several methodologies, and finding a reference somewhere would probably be good.
\end{comment}

\section{Research methods}

In order to answer the research goals and achieve the goal of this thesis, a proof of concept method was developed and implemented by the author. In order to have an implementation the new method could be tested against, an existing code base was used as chosen as a starting point. The most fitting existing method was \gls{flex} \cite{wickramanayakeFLEXFaithfulLinguistic2019}. 

The reason why this method was chosen was that it proposes a novel method to associate features responsible for the decisions in a \gls{cnn} model with words. This way it can generate intuitive, descriptive and faithful explanations that annotate objects in an image. These post-hoc justifications for classification using natural language is achieved using a \gls{cnn} for extracting features in images and two stacked \glspl{lstm} to extract features from text descriptions during training phase, and use this knowledge and \glspl{lstm}, in conjunction with image features to give faithful descriptions during testing, when no text descriptions is supplied with the image.

