\label{sec:4_experiments_and_results}

\begin{comment}
It is important to “verify” your solutions and ideas with an evaluation. This should be described in this chapter.

INTRO: What will you look at in this chapter and why? Again, point back to the summary in the last chapter, and say that here you want to experiment with and evaluate the proposed “solution”.

MIDDLE SECTIONS: Explain your experiments and evaluations. Include a detailed description of the data you have used, which metrics you include. It is nice to discuss what the results also mean (in general and in the context of your problem statement), not only what you can observe. Also, try to explain WHY the results turn out as they do. You are a researcher and should try to understand why things happen, not only observe what happens.

DISCUSSION: (some put this as a separate chapter before the conclusion depending on the length of it) It is often also nice to discuss the results in a broader setting, trying to generalize the results beyond the specific case study and selected data. This is often done in a separate discussion section at the end - or if it is a lot to discuss, as a separate chapter. Here, one can also typically include a discussion of challenges and pitfalls experienced, etc.

SUMMARY: As above, the summary section should summarize your achievements, results, etc. Briefly conclude what they mean, and what you have learned. NO need to lead to the next chapter - conclusion.
\end{comment}


% INTRO:



% MIDDLE SEXTIONS:



% DISCUSSION:


    % Using more modern CV-methods, like YOLO, instead of gradients of CNN.


% SUMMARY: